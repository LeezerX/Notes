\subsection*{萌芽}6

\begin{enumerate}
    \item 1654年7月29日,法国职业赌徒骑士D.Mere向Pascal提出一系列问题,史称Mere问题
    \item Pascal与Fermat通信交流这些问题
    \item 荷兰数学家Huggens听说了Fermat与Pascal的交流,便参与进去,于1657年出版了《论赌博中的计算》.\textbf{概率论从此诞生}
\end{enumerate}

\subsection*{发展}

\begin{enumerate}
    \item Jacob Bernoulli 于1713出版《猜度术》,首次提出以伯努利定理著称的\textbf{极限定理}
    \item A.DE Maivre 于1773年以及 Gauss 于1809年独自引进正态分布
    \item Laplace 于1812年出版《概率基础的分析理论》,使以往零散的结果系统化。
    \item 俄国数学家切比雪夫对极限理论作出重要贡献
    \item 1899年 J.Bertrand 提出贝特朗悖论
    \item D.Hilbort 在1900年的世界数学大会上提出了建立概率公理系统的问题
\end{enumerate}

\subsection*{突破}
\begin{enumerate}
    \item 1933年,前苏联数学家柯尔莫戈罗夫出版了《概率论基础》,建立了概率论的公理化定义。
\end{enumerate}